
This second volume of \myquote{Basic Analysis} is meant to
be a seamless continuation.  The chapter numbers start where the
first volume left off.  The book started with my notes for a second-semester
undergraduate analysis at University of Wisconsin---Madison in 2012, which
I taught more or less with Rudin's book.  Some of the
material and some of the proofs are similar to Rudin, though I try
to provide more detail and context.
In 2016, I taught a second-semester
undergraduate analysis at Oklahoma State University, modifying
and cleaning up the notes, this time using them as the main text.
I have since taught the course several more times, adding chapter
\ref{approx:chapter} (originally written for the Wisconsin course),
and making many other smaller improvments.

I plan to eventually add a few more topics.
I will try to preserve the numbering in subsequent editions as always.
The new topics planned would add chapters onto the end of the
book, or add sections to end of existing chapters, and I will try as hard as
possible to leave exercise numbers unchanged.

For the most part, this second volume
depends on the non-optional parts of volume I\@,
while some of the optional parts are also used.
Higher order derivatives (but not Taylor's theorem
itself) are used
in \ref{sec:mvhighordders}, \ref{sec:pathind},
\ref{sec:mvgreenstheorem}.  Exponentials, logarithms, and improper integrals are
used in a few examples and exercises, and they are heavily used in
\chapterref{approx:chapter}.
%It is entirely possible to avoid using all
%the optional parts from volume I\@.

%This book is not necessarily the entire second-semester course,
%though it should have enough material for an entire semester if need be.
An alternate plan for a
two-semester course is that some bits of the
first volume, such as metric spaces, are
covered in the second semester, while some of the optional topics of volume
I are covered in the first semester.  Leaving metric spaces for the second
semester makes the second
semester the \myquote{multivariable} part of the course.

Several possibilities for things to cover after metric spaces,
depending on time are:
\begin{enumerate}[1)]
\item
%\volIref{Chapter \ref*{vI-ms:chapter} from volume I}{Chapter \ref{ms:chapter}},
\ref{sec:vectorspaces}--\ref{sec:svinvfuncthm},
\ref{sec:rirect}--\ref{sec:jordansets}, \ref{sec:mvchangeofvars}
(multivariable calculus, focus on multivariable integral).
\item
%\volIref{Chapter \ref*{vI-ms:chapter} from volume I}{Chapter \ref{ms:chapter}},
Chapter \ref{pd:chapter}, chapter \ref{path:chapter},
%\ref{sec:vectorspaces}--\ref{sec:mvhighordders},
%\ref{sec:diffunderint}--\ref{sec:pathind},
\ref{sec:rirect} and \ref{sec:iteratedints}
(multivariable calculus, focus on path integrals).
\item Chapters \ref{pd:chapter}, \ref{path:chapter}, and
\ref{mi:chapter}
(multivariable calculus, path integrals, multivariable integrals).
\item
%\volIref{Chapter \ref*{vI-ms:chapter} from volume I}{Chapter \ref{ms:chapter}},
Chapters \ref{pd:chapter}, (maybe \ref{path:chapter}), and \ref{approx:chapter}
(multivariable differential calculus, some advanced analysis).
\item
Chapter \ref{pd:chapter}, chapter \ref{path:chapter}, 
\ref{sec:complexnums},
\ref{sec:arzelaascoli},
\ref{sec:stoneweier}
(a simpler variation of the above).
\end{enumerate}

%When I ran the course at OSU\@, I covered the first book minus metric spaces
%and a couple of optional sections in the first semester.
%Then, in the second semester, I covered
%most of what I skipped from volume I\@, including metric spaces, and
%took option 2) above.
