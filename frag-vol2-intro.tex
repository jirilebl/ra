
This book is the continuation of \myquote{Basic Analysis}.  The book is meant to
be a seamless continuation, so the chapters are numbered to start where the
first volume left off.  The book started with my notes for a second-semester
undergraduate analysis at University of Wisconsin---Madison in 2012, where I
used my notes together with Rudin's book.  The choice of some of the
material and many of the proofs are very similar to Rudin, though I do try
to provide more detail and context.
In 2016, I taught a second-semester
undergraduate analysis at Oklahoma State University and heavily
modified and cleaned up the notes, this time using them as the main text.
In 2018, I taught this course again, this time adding chapter
\ref{approx:chapter} (which I originally wrote for the Wisconsin course).

I plan on eventually adding some more topics.
I will try to
preserve the current numbering in subsequent editions as always.  The new
topics I have planned would add chapters onto the end of the
book, or add sections to end of existing chapters, and I will try as hard as
possible to leave exercise numbers unchanged.

For the most part, this second volume
depends on the non-optional parts of volume I\@.
Of the optional parts, higher order derivatives (but not Taylor's theorem
itself) are used
in \ref{sec:mvhighordders}, \ref{sec:pathind},
\ref{sec:mvgreenstheorem}.  Exponentials, logarithms, improper integrals are
used a few times in examples and exercises, and they are heavily used in
\chapterref{approx:chapter}.
%It is entirely possible to avoid using all
%the optional parts from volume I\@.

This book is not necessarily the entire second-semester course,
though it should have enough material for an entire semester if need be.
What I had in mind for a two-semester course is that some bits of the
first volume, such as metric spaces, are
covered in the second semester, while some of the optional topics of volume
I are covered in the first semester.  Leaving metric spaces for the second
semester makes more sense as then the second
semester is the \myquote{multivariable} part of the course.

Another possibility for a faster course
is to leave out some of the optional parts, go quicker in the first semester
including metric spaces and then arrive at \chapterref{approx:chapter}.

Several possibilities for things to cover after metric spaces,
depending on time are:
\begin{enumerate}[1)]
\item
%\volIref{Chapter \ref*{vI-ms:chapter} from volume I}{Chapter \ref{ms:chapter}},
\ref{sec:vectorspaces}--\ref{sec:svinvfuncthm},
\ref{sec:rirect}--\ref{sec:jordansets}, \ref{sec:mvchangeofvars}.
\item
%\volIref{Chapter \ref*{vI-ms:chapter} from volume I}{Chapter \ref{ms:chapter}},
Chapter \ref{pd:chapter}, chapter \ref{path:chapter},
%\ref{sec:vectorspaces}--\ref{sec:mvhighordders},
%\ref{sec:diffunderint}--\ref{sec:pathind},
\ref{sec:rirect} and \ref{sec:iteratedints}.
\item Chapters \ref{pd:chapter}, \ref{path:chapter}, and
\ref{mi:chapter}.
\item
%\volIref{Chapter \ref*{vI-ms:chapter} from volume I}{Chapter \ref{ms:chapter}},
Chapters \ref{pd:chapter}, (maybe \ref{path:chapter}), and \ref{approx:chapter}.
\item
Chapter \ref{pd:chapter}, chapter \ref{path:chapter}, 
\ref{sec:complexnums},
\ref{sec:arzelaascoli},
\ref{sec:stoneweier}.
\end{enumerate}

%When I ran the course at OSU\@, I covered the first book minus metric spaces
%and a couple of optional sections in the first semester.
%Then, in the second semester, I covered
%most of what I skipped from volume I\@, including metric spaces, and
%took option 2) above.
