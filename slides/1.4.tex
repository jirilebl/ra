\documentclass[10pt,aspectratio=149]{beamer}

% All the boilerplate is in raslides.sty
% Note that this also pulls in a custom vogtwidebar.sty
\usepackage{raslides}

\author{Ji\v{r}\'i Lebl}

\institute[OSU]{%
Departemento pri Matematiko de Oklahoma {\^S}tata Universitato}

\title{BA: 1.4}

\date{}

\begin{document}

\begin{frame}
\titlepage
\end{frame}

\begin{frame}
For $a < b$, define the intervals:

\medskip
\pause

$[a,b] \coloneqq \{ x \in \R : a \leq x \leq b \}$
\quad (closed interval)

\pause

$(a,b) \coloneqq \{ x \in \R : a < x < b \}$
\quad (open interval)

\pause

$(a,b] \coloneqq \{ x \in \R : a < x \leq b \}$
\quad (half-open interval)

\pause

$[a,b) \coloneqq \{ x \in \R : a \leq x < b \}$
\quad (half-open interval)

\pause
\medskip

Those were bounded intervals.  Also define unbounded intervals:

\medskip
\pause

$[a,\infty) \coloneqq \{ x \in \R : a \leq x \}$

\pause

$(a,\infty) \coloneqq \{ x \in \R : a < x \}$

\pause

$(-\infty,b] \coloneqq \{ x \in \R : x \leq b \}$

\pause

$(-\infty,b) \coloneqq \{ x \in \R : x < b \}$

\pause

$(-\infty,\infty) \coloneqq \R$

\pause

\begin{proposition}
A set $I \subset \R$ is an interval if and only if
$I$ contains at least 2 points and
for all $a,c \in I$ and $b \in \R$ such that $a < b < c$, we have $b \in I$.
\end{proposition}

\pause

Proof is an exercise.

\end{frame}

\begin{frame}
Set theoretically intervals have the same size.

\medskip
\pause

$f(x)\coloneqq 2x$ ~is a bijection from $[0,1]$ to $[0,2]$

\medskip
\pause

$f(x) \coloneqq \tan(x)$ ~is a bijection from $(-\nicefrac{\pi}{2},\nicefrac{\pi}{2})$ to $\R$

\medskip
\pause

Bijection from $[0,1]$ to $(0,1)$ is harder but possible.

\medskip
\pause

We saw that uncountable sets exist: e.g., $\sP(\N)$.

\pause

\begin{theorem}[Cantor]
$\R$ is uncountable.
\end{theorem}

\pause

So any interval is uncountable.

\medskip
\pause

Also, the set $\R \setminus \Q$ is uncountable.

\medskip
\pause

We'll give essentially Cantor's original 1874 proof.

\end{frame}

\begin{frame}
\textbf{Proof:}
Suppose $X \subset \R$ is countably infinite such that

for every pair of real numbers $a < b$, there is an $x \in X$ such that
$a < x < b$.

\pause

Were $\R$ countable, we could take $X=\R$.  \qquad We'll show $X \subsetneq \R$.

\medskip
\pause

Write $X = \{ x_1, x_2, x_3, \ldots \}$ \quad ($X$ is countably infinite).

\pause
We construct two sequences
$a_1,a_2,a_3,\ldots$ and
$b_1,b_2,b_3,\ldots$

\pause
Let $a_1 \coloneqq x_1$ and $b_1 \coloneqq x_1+1$.

\pause
Note that $a_1 < b_1$ and $x_1 \notin (a_1,b_1)$.

\pause
Suppose for some $k > 1$, 
$a_{j}$ and $b_{j}$ have been defined for $j=1,2,\ldots,k-1$,

\pause
for each such $j$, suppose $x_\ell \notin (a_{j},b_{j})$ for $\ell=1,2,\ldots,j$,

\pause
and suppose
$a_1 < a_2 < \cdots < a_{k-1} < b_{k-1} < \cdots < b_2 < b_1$.

\medskip
\pause

Set $a_k \coloneqq x_n$, where $n$ is the smallest $n \in \N$
such that $x_n \in (a_{k-1},b_{k-1})$.

\pause
$x_n$ exists by assumption on $X$.
\quad $n \geq k$ by assumption on $(a_{k-1},b_{k-1})$.

\pause
Define $b_k$ to be any real number in $(a_k,b_{k-1})$.

\medskip
\pause

Note $a_{k-1} < a_k < b_k < b_{k-1}$, and

\pause
$x_k \notin (a_{k},b_{k})$ and hence
$x_j \notin (a_{k},b_{k})$ 
for $x_j$ for $j=1,2,\ldots,k$.

\pause
The two sequences are now defined.

\end{frame}

\begin{frame}
We have two sequences such that

for each $j$, suppose $x_\ell \notin (a_{j},b_{j})$ for $\ell=1,2,\ldots,j$,

and 
$a_1 < a_2 < \cdots < a_{k} < b_{k} < \cdots < b_2 < b_1$ (for every $k$).

\medskip
\pause

\textbf{Claim:} \emph{$a_n < b_m$ for all $n$ and $m$ in $\N$.}

\pause
\textbf{Proof of claim:}
Suppose $n < m$ \pause \wthus
$a_n < a_{n+1} < \cdots < a_{m-1} < a_m < b_m$.

\pause
Similarly for $n > m$.  The claim follows.

\medskip
\pause

Let $A \coloneqq \{ a_n : n \in \N \}$ and $B \coloneqq \{ b_n : n \in \N \}$.

\pause
By the claim, \quad $\sup\, A \leq \inf\, B$.

\pause
Define $y \coloneqq \sup\, A$.

\pause
$y \notin A$: If $y=a_n$ for some $n$, then $y < a_{n+1}$, which is impossible.

\pause
$y \notin B$: Similar.

\pause
So $a_n < y < b_n$ for all $n \in \N$ \quad or \quad $y \in (a_n,b_n)$ for
all $n \in \N$.

\pause
For every $n \in \N$, $x_n \not\in (a_n,b_n)$, so $y \not= x_n$.

\pause
\thus \quad $y \notin X$ \pause \wthus $X \subsetneq \R$ \pause \wthus $\R$
is uncountable.
\qed
\end{frame}

\begin{frame}
\textbf{Remark:}
Cantor's paper was (perhaps oddly) about algebraic numbers:

\medskip
\pause

$x \in \R$ is \emph{algebraic} if $x$ is a root of a polynomial with
integer coefficients:

$a_n x^n + a_{n-1} x^{n-1}  + \cdots
+ a_1 x + a_0 = 0$ \quad where $a_0,a_1,\ldots,a_n \in \Z$.

\medskip
\pause

There are only countably many algebraic numbers (exercise).

\medskip
\pause

Cantor's theorem shows $\exists$ non-algebraic (transcendental)
numbers.

\end{frame}

\end{document}
