\documentclass[10pt,aspectratio=149]{beamer}

% All the boilerplate is in ac1slides.sty
% Note that this also pulls in a custom vogtwidebar.sty
\usepackage{ac1slides}

\author{Ji\v{r}\'i Lebl}

\institute[OSU]{%
Departemento pri Matematiko de Oklahoma {\^S}tata Universitato}

\title{BA: 5.2}

\date{}

\begin{document}

\begin{frame}
\titlepage
\end{frame}

\begin{frame}
\begin{lemma}[Additivity]
Suppose $a < b < c$ and $f \colon [a,c] \to \R$ is a bounded function.
\pause
Then
\begin{equation*}
\underline{\int_a^c} f
=
\underline{\int_a^b} f
+
\underline{\int_b^c} f
\pause
\quad \text{and} \quad
\overline{\int_a^c} f
=
\overline{\int_a^b} f
+
\overline{\int_b^c} f .
\end{equation*}
\end{lemma}

\pause
\textbf{Proof:}
Let $P_1 = \{ x_0,x_1,\ldots,x_k \}$ be a partition of $[a,b]$,

\pause
and $P_2 = \{ x_k, x_{k+1}, \ldots, x_n \}$ of $[b,c]$,

\pause
\thus~ $P \coloneqq P_1 \cup P_2 = \{ x_0, x_1, \ldots, x_n \}$ is
a partition of $[a,c]$.

\pause
\medskip

\quad $\displaystyle
L(P,f)
\pause
=
\sum_{i=1}^n m_i \Delta x_i
\pause
=
\sum_{i=1}^k m_i \Delta x_i
+
\sum_{i=k+1}^n m_i \Delta x_i
\pause
=
L(P_1,f) + L(P_2,f)$.

\pause
\medskip

$\sup$ of RHS over all $P_1$ and $P_2$ is like
$\sup$ of LHS over all $P$ s.t. $b \in P$.

\pause
\medskip

If $P=Q \cup \{ b \}$, then $P$ is a refinement of $P$, so $L(Q,f) \leq
L(P,f)$.

\pause
\medskip

\thus~ $\sup L(P,f)$ over all $P$ with $b \in P$ is sufficient
to get $\sup L(P,f)$ over all $P$.

\end{frame}

\begin{frame}

$\displaystyle
\underline{\int_a^c} f
\pause
=
\sup \, \bigl\{ L(P,f) : P \text{ a partition of } [a,c] \bigr\}
$

\pause
\medskip

\qquad
$\displaystyle
=
\sup \, \bigl\{ L(P,f) : P \text{ a partition of } [a,c], b \in P \bigr\}
$

\pause
\medskip

\qquad
$\displaystyle
=
\sup \, \bigl\{ L(P_1,f) + L(P_2,f) :
P_1 \text{ a partition of } [a,b], P_2 \text{ a partition of } [b,c] \bigr\}
$

\pause
\medskip

\qquad
$\displaystyle
=
\sup \, \bigl\{ L(P_1,f) : P_1 \text{ a partition of } [a,b] \bigr\}
$

\medskip

\qquad\qquad
$\displaystyle
+
\sup \, \bigl\{ L(P_2,f) : P_2 \text{ a partition of } [b,c] \bigr\}
$

\pause
\medskip

\qquad
$\displaystyle
=
\underline{\int_a^b} f + \underline{\int_b^c} f$.

\pause
\medskip

The upper integral argument is analogous (see book).
\qed

\end{frame}

\begin{frame}

\begin{proposition}[Additivity]
Let $a < b < c$.\quad $f \colon [a,c] \to \R$ is Riemann integrable
if and only if $f$ is Riemann integrable on $[a,b]$ and $[b,c]$.
\pause
If
$f$ is Riemann integrable, then
\begin{equation*}
\int_a^c f
=
\int_a^b f
+
\int_b^c f .
\end{equation*}
\end{proposition}

\pause
\textbf{Proof:}
Suppose $f \in \sR[a,c]$, then 
$\overline{\int_a^c} f = 
\underline{\int_a^c} f = 
\int_a^c f$.

\pause
\medskip

The lemma \wthus
$\displaystyle
\int_a^c f
\pause
=
\underline{\int_a^c} f
\pause
 =
\underline{\int_a^b} f + \underline{\int_b^c} f
\pause
 \leq
\overline{\int_a^b} f + \overline{\int_b^c} f
\pause
 =
\overline{\int_a^c} f
\pause
 =
\int_a^c f$.

\pause
\medskip

\thus \quad $\displaystyle
\underline{\int_a^b} f + \underline{\int_b^c} f
=
\overline{\int_a^b} f + \overline{\int_b^c} f$.

\pause
\medskip

$\displaystyle\underline{\int_a^b} f \leq \overline{\int_a^b} f$
~and~
$\displaystyle\underline{\int_b^c} f \leq \overline{\int_b^c} f$
\pause
\wthus
$\displaystyle
\underline{\int_a^b} f
=
\overline{\int_a^b} f$
~and~
$\displaystyle
\underline{\int_b^c} f
=
\overline{\int_b^c} f
$.

\pause
\medskip

\thus \quad $f$ is Riemann integrable on $[a,b]$ and $[b,c]$ and
$\int_a^c f
=
\int_a^b f
+
\int_b^c f$.
\end{frame}

\begin{frame}

Now assume $f$ is Riemann integrable on $[a,b]$ and on $[b,c]$.

\pause
\medskip

\quad
$\displaystyle
\underline{\int_a^c} f
\pause
=
\underline{\int_a^b} f + \underline{\int_b^c} f
\pause
=
\int_a^b f + \int_b^c f
\pause
=
\overline{\int_a^b} f + \overline{\int_b^c} f
\pause
=
\overline{\int_a^c} f$.

\pause
\medskip

\thus \quad $f$ is Riemann integrable on $[a,c]$ and
$\int_a^c f
=
\int_a^b f
+
\int_b^c f$.
\qed


\pause
\bigskip

\begin{corollary}
If $f \in \sR[a,b]$ and
$[c,d] \subset [a,b]$, then
the restriction $f|_{[c,d]}$ is in $\sR[c,d]$.
\end{corollary}

\pause
\textbf{Proof:} Exercise.

\end{frame}

\begin{frame}

\begin{proposition}[Linearity]
Let $f$ and $g$ be in $\sR[a,b]$ and $\alpha \in \R$.
\begin{enumerate}[(i)]
\item
\pause
$\alpha f$ is in $\sR[a,b]$ and
$\int_a^b \alpha f(x) \,dx = \alpha \int_a^b f(x) \,dx$.
\item
\pause
$f+g$ is in $\sR[a,b]$ and
$\int_a^b \bigl( f(x)+g(x) \bigr) \,dx = 
\int_a^b f(x) \,dx 
+
\int_a^b g(x) \,dx$.
\end{enumerate}
\end{proposition}

\pause
\textbf{Proof:}
(i) Suppose $\alpha \geq 0$.
\pause
\quad
Let $P$ be a partition of $[a,b]$, \quad $m_i$ as usual.

%$m_i \coloneqq \inf \bigl\{ f(x) : x \in [x_{i-1},x_i] \bigr\}$ as usual.

\pause
$\inf \bigl\{ \alpha f(x) : x \in [x_{i-1},x_i] \bigr\}
\pause
=
\alpha \inf \bigl\{ f(x) : x \in [x_{i-1},x_i] \bigr\}
\pause
= \alpha m_i$.

\pause
\medskip

$\displaystyle
L(P,\alpha f)
\pause
=
\sum_{i=1}^n \alpha m_i \Delta x_i
\pause
=
\alpha \sum_{i=1}^n m_i \Delta x_i
\pause
=
\alpha L(P,f)$.

\pause
\medskip

Similarly,
$U(P,\alpha f) = \alpha U(P,f)$.

\pause
\medskip

$%\displaystyle
\underline{\int_a^b} \alpha f(x)\,dx
\pause
=
\sup \, \bigl\{ L(P,\alpha f) : P \text{ a partition of } [a,b] \bigr\}
$

\pause
\qquad
$
=
\sup \, \bigl\{ \alpha L(P,f) : P \text{ a partition of } [a,b] \bigr\}
$

\pause
\qquad
$
=
\alpha \,
\sup \, \bigl\{ L(P,f) : P \text{ a partition of } [a,b] \bigr\}
\pause
=
\alpha
\underline{\int_a^b} f(x)\,dx
$.

\pause
Similarly,
$
\overline{\int_a^b} \alpha f(x)\,dx
=
\alpha
\overline{\int_a^b} f(x)\,dx$.
\pause
\wthus
(i) follows for $\alpha \geq 0$.

\end{frame}

\begin{frame}

\textbf{Exercise:} Finish proof of (i) ($\alpha < 0$):

\pause
Show $-f$ is Riemann integrable and
$\int_a^b - f(x)\,dx =
-
\int_a^b f(x)\,dx$.

\pause
\medskip

\textbf{Exercise:} Prove (ii).

\medskip
\qed

\pause
\medskip

(ii) is not as trivial as may seem,
\pause
see

\begin{proposition}
Let $f \colon [a,b] \to \R$ and $g \colon [a,b] \to \R$ be bounded
functions.  Then
\begin{equation*}
%\overline{\int_a^b} \bigl(f(x)+g(x)\bigr)\,dx \leq
%\overline{\int_a^b}f(x)\,dx+\overline{\int_a^b}g(x)\,dx
\overline{\int_a^b} (f+g) \leq \overline{\int_a^b}f+\overline{\int_a^b}g
,
\qquad
\text{and}
\qquad
\underline{\int_a^b} (f+g) \geq \underline{\int_a^b}f+\underline{\int_a^b}g
%\underline{\int_a^b} \bigl(f(x)+g(x)\bigr)\,dx \geq
%\underline{\int_a^b}f(x)\,dx+\underline{\int_a^b}g(x)\,dx
.
\end{equation*}
\end{proposition}

\pause
\textbf{Proof:} Exercise.

\end{frame}

\begin{frame}

\begin{proposition}[Monotonicity]
Let $f \colon [a,b] \to \R$ and $g \colon [a,b] \to \R$ be
bounded, and $f(x) \leq g(x)$
$\forall$ $x \in [a,b]$.
\pause
Then
\begin{equation*}
\underline{\int_a^b} f 
\leq
\underline{\int_a^b} g 
\qquad \text{and} \qquad
\overline{\int_a^b} f 
\leq
\overline{\int_a^b} g .
\end{equation*}
\pause
Moreover, if $f$ and $g$ are in $\sR[a,b]$, then
$\displaystyle
\int_a^b f 
\leq
\int_a^b g$.
\end{proposition}

\pause
\textbf{Proof:}
Let $P = \{ x_0, x_1, \ldots, x_n \}$ be a partition of $[a,b]$.

\pause
Let
$m_i \coloneqq \inf \, \bigl\{ f(x) : x \in [x_{i-1},x_i] \bigr\}$
~and~
$\widetilde{m}_i \coloneqq \inf \, \bigl\{ g(x) : x \in [x_{i-1},x_i]
\bigr\}$.

\pause
$f(x) \leq g(x)$ for all $x$ \wthus $m_i \leq \widetilde{m}_i$.

\pause
\medskip

$\displaystyle
L(P,f)
\pause
=
\sum_{i=1}^n m_i \Delta x_i
\pause
\leq
\sum_{i=1}^n \widetilde{m}_i \Delta x_i
\pause
=
L(P,g)$
\pause
\wwthus
$\underline{\int_a^b} f 
\leq
\underline{\int_a^b} g$.

\pause
\medskip

Similarly,
$\overline{\int_a^b} f 
\leq
\overline{\int_a^b} g$.

\pause
\medskip

If $f$ and $g$ are integrable, \wthus
$\int_a^b f 
\leq
\int_a^b g$.
\qed

\end{frame}

\begin{frame}

\begin{lemma}
If $f \colon [a,b] \to \R$ is continuous,
then $f \in \sR[a,b]$.
\end{lemma}

\pause
\textbf{Proof:}
$[a,b]$ is closed \& bounded
\pause
\hfill\thus\hfill $f$ is bounded \&
uniformly continuous.

\pause
Given $\epsilon > 0$, find $\delta > 0$ s.t.
$\abs{x-y} < \delta$ implies $\abs{f(x)-f(y)} < \frac{\epsilon}{b-a}$.

\pause
Let $P = \{ x_0, x_1, \ldots, x_n \}$
be a partition s.t. $\Delta x_i < \delta$ for all $i$.

%For example,
%take $n$ such that $\frac{b-a}{n} < \delta$, and
%let $x_i \coloneqq \frac{i}{n}(b-a) + a$.

\pause
\medskip

$\forall$ $x, y \in [x_{i-1},x_i]$, ~~
$\abs{x-y} \leq \Delta x_i < \delta$,
\pause
\wthus
$
f(x)-f(y)
\pause
\leq
\abs{f(x)-f(y)}
\pause
<
\frac{\epsilon}{b-a}$.

\pause
\medskip

$f$ continuous so 
$f$ attains a maximum on $[x_{i-1},x_i]$ at $x$ and
a minimum at $y$.

\pause
\thus \quad $f(x) = M_i$ and $f(y) = m_i$ ~~~(usual notation).

\pause
\medskip

\thus\quad
$M_i-m_i = f(x)-f(y) < \frac{\epsilon}{b-a}$.

\pause
\medskip

\thus\quad
$\displaystyle
\overline{\int_a^b} f - 
\underline{\int_a^b} f 
\pause
\leq
U(P,f) - L(P,f)
\pause
=
\left(
\sum_{i=1}^n
M_i \Delta x_i
\right)
-
\left(
\sum_{i=1}^n
m_i \Delta x_i
\right)
$

\pause
\medskip

\qquad\qquad
$\displaystyle
=
\sum_{i=1}^n
(M_i-m_i) \Delta x_i
\pause
<
\frac{\epsilon}{b-a}
\sum_{i=1}^n
\Delta x_i
\pause
=
\frac{\epsilon}{b-a} (b-a)
= \epsilon$.

\pause
\medskip

$\epsilon > 0$ was arbitrary
\pause
\wthus
$\overline{\int_a^b} f = \underline{\int_a^b} f$
\pause
\wthus
$f$ is Riemann integrable.
\qed

\end{frame}

\begin{frame}

It is not hard to prove that functions with finitely many discontinuities
are also integrable (we skip the proof),
\pause
although a lot more is true.

\pause
\medskip

Also,

\begin{proposition}
Let $f \colon [a,b] \to \R$ be a monotone function.  Then $f \in \sR[a,b]$.
\end{proposition}

\pause
\textbf{Proof:} Exercise.

\pause
\medskip

Functions $h$ of the form $h=f-g$ where $f$ and $g$ are monotone are said to
be of \emph{bounded variation}.

\pause
\medskip

Such functions are Riemann integrable.

\end{frame}

\begin{frame}

\textbf{Exercise:}
Prove the \emph{mean value theorem for integrals}:
If $f \colon [a,b] \to \R$ is continuous, then there exists
a $c \in [a,b]$ such that $\int_a^b f = f(c)(b-a)$.

\pause
\medskip

\textbf{Exercise:}
Let $f \colon [a,b] \to \R$ be a continuous function such that $f(x) \geq 0$
for all $x \in [a,b]$ and $\int_a^b f = 0$.  Prove that $f(x) = 0$
for all $x$.

\pause
\medskip

\textbf{Exercise} (Challenging)\textbf{:}
Suppose $f \in \sR[a,b]$, then the function that takes $x$ to
$\abs{f(x)}$ is also Riemann integrable on $[a,b]$.
\pause
Then show
\[
\abs{\int_a^b f(x)\,dx} \leq \int_a^b \abs{f(x)}\,dx .
\]

\end{frame}

\end{document}
