
\subsubsection{Integral form of the remainder in Taylor's theorem}

In \sectionref{sec:taylor} we looked at a function $f \colon [a,b] \to \R$
with $n+1$ derivatives and for a point $x_0 \in [a,b]$ we wrote
\begin{equation*}
f(x)=P_{n}^{x_0}(x)+R_{n}^{x_0}(x) ,
\end{equation*}
where where $P_n^{x_0}$ is the $n$th \myindex{Taylor polynomial} for 
$f$ at $x_0$ and 
$R_{n}^{x_0}(x)$ is the \myindex{remainder term}.
Let us explore another form of the remainder term.

\begin{thm}
Suppose $f \colon [a,b] \to \R$ is a function with $n+1$ continuous
derivatives on $[a,b]$.
Given distinct points $x_0$ and $x$ in $[a,b]$
\begin{equation*}
f(x)=P_{n}^{x_0}(x)+
\int_{x_0}^x
\frac{f^{(n+1)}(t)}{n!}{(x-t)}^{n} ~dt .
\end{equation*}
\end{thm}

\begin{proof}
FIXME
\end{proof}
